\documentclass[../Main.tex]{subfiles}
\begin{document}
\section{Spider}
The Spider downloads repositories from a codebase (currently only GitLab and GitHub) and then gathers \texttt{AuthorData} from the downloaded repository. This \texttt{AuthorData} is then sent to the controller so other components can use this data. The downloaded repository stays locally on disk so the parser can parse the files of that repository.
\subsection{RunSpider}
Starting the process of downloading and gathering \texttt{AuthorData} is done in several steps. First, \texttt{RunSpider::setupSpider} sets up a Spider specific to a given codebase, with the specified number of threads. To start downloading, this same Spider object is needed.\\

The function \texttt{RunSpider::downloadRepo} clones a repository to the specified location. Further, the function \texttt{RunSpider::getAuthors} creates \texttt{.meta} files containing all the authors of the files currently in the downloads folder. To switch from one tag to another, \texttt{RunSpider::updateVersion} is used. This method returns a list of unchanged files and deletes these from the downloads folder (to prevent them from being parsed). Currently the program checks if the URL is either a GitHub or a GitLab URL and if that is the case, a \texttt{GitSpider} is created. The GitSpider class is a subclass of the Spider class. If you want to add a different type of Spider for a different type of website, you can add a new regex check to the \texttt{RunSpider::getSpider} method which returns a specific Spider subclass in charge of handling that website.
\subsection{Spider class}
The spider class is responsible for downloading projects from a codebase. It is an abstract class which has a few methods already predefined. Having a Spider superclass this way allows the \texttt{RunSpider} to work with arbitrary Spider implementations without having to worry about how the data is retrieved. 
\subsection{Git class}
In the case of the GitSpider class, downloading the source is done with help of a Git class. To download a git repository a \texttt{git clone} command is called. The \texttt{git clone} command is set up to perform a sparse checkout which allows us to only download the files with certain extension types. This way we ignore all files that cannot be parsed by the parser.
\subsection{Tags}
The Git class can also get a specific tag from a git project. A tag is a commit that the developer tagged as important. These tags can be retrieved by calling \texttt{runSpider:getTags} after a repository has been downloaded. The Git download function is given a 'tag' argument and a 'nextTag' argument. The 'nextTag' argument indicates which tag to download. The 'tag' argument is used for comparison. The Spider expects the oldest tag to be downloaded first. Afterwards you can always give the next newest tag as 'nextTag' and the old tag as 'tag'. Then the spider checks which files changed between the tags and throws away the unchanged files as they have already been parsed previously.
\subsection{Author data}
The \texttt{AuthorData} data structure consists of a vector of \texttt{CodeBlock} data structures for every file in the repository. A \texttt{CodeBlock} consists of a starting line variable, a \# of lines variable and a \texttt{CommitData} data structure. All lines from the starting line till the starting line plus the \# of lines belong to that \texttt{CommitData}. The \texttt{CommitData} data structure contains data such as who the author is, what the email address of the author is and some other miscellaneous things. This data is useful to check which author wrote which parts of a file.
\subsection{Branches}
\texttt{RunSpider::runSpider} also allows has an argument to define which branch should be downloaded of the repository. By default, only the branch that is defined as the ''Default branch" in Git in downloaded, but by using this argument it's possible to download different branches if that is desirable.
\subsection{Collecting author data}
In the GitSpider class, author data is collected using \texttt{git blame}. Every file in the downloading repository gets a \texttt{git blame} command called on it and the output is written to a .meta file. After every file has been blamed, the blame data of all the .meta files gets parsed by the Git class and then stored in an \texttt{AuthorData} data structure.
\subsection{Error handling}
There are a few errors that can happen during the spidering process, for example being unable to download any files because there is no internet connection or there being some strange data among the blame data. In some cases only a warning is given, for example if the spider is unable to parse certain blame data, then that blame data is skipped and the spider continues as usual. This approach is sometimes not possible, for example if the spider fails to download anything. In this case it throws an error and stops running, the controller has to decide how to proceed afterwards.
\end{document}