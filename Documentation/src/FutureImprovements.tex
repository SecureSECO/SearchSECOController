\documentclass[./Main.tex]{subfiles}
\begin{document}

\section{Future improvements}
Here, we discuss observations we made that could be improved in the future.

\begin{itemize}
    \item The custom parser is quite a bit (30 times) slower than the srcML parser. We could improve this by constructing a more simplified grammar than the general one provided by ANTLR.
    \item In the database API, multiple methods use similar code leading to much code duplication. Think for example of the single query thread methods. This could be fixed by adding a bit of abstraction to the code. We have also partially applied abstraction already by means of template functions, but other approaches may also be possible.
    \item In the database API, some components (i.e., ConnectionHandler, Networking and RAFTConsensus) are not (fully) tested due to the fact that the tests did not pass consistently. It would be useful to fix these tests to obtain a better code coverage.
    \item The unit tests in general do not cover all edge cases. It would be good to improve this.
    \item The custom parser needs to convert files to UTF-8 since that is the format ANTLR requires, currently there is some code which converts ANSI to UTF-8 and seems to also work on some other formats. It is not known what happens when the parser encounters a format/character that it can't work with, hopefully this will result in an error in the conversion which would be catched and handled. It could however also be send to ANTLR causing a crash.
    \item Communication between controller and database-api, as well as between multiple database-api's, is currently unencrypted in plain text. It would be trivial to send data to the database-api from any system, and possibly to find and exploit vulnerabilities in the database-api to hack a system running it.
    \item The contact between the database-api and the database as well as the different databases is not encrypted.
    \item The contact between the database-api and the database is not authenticated, so anyone would be able to execute queries on the database quite easily.
    \item SrcML seems to not recognise all valid code, when encountering something it doesn't recognise there is a chance that the rest of the file won't be parsed properly either. Exact cases that srcML doesn't recognise are not known to us.
    \item The job queue is currently saved in a Cassandra database, this seems to work fine at the moment, but since the Cassandra database handles deletes by inserting tombstones (leading to longer lookup times since all tombstones will be found first) this could lead to problems when scaling up the system.
    \item Sometimes a job can get stuck. There is a timeout implemented to combat this, but even with this timeout it can get stuck.
    \item Some of the timeouts are still a bit short or long, so it would be good to take another good look into tweaking the timeouts.
    \item To check whether a project has already been uploaded the controller asks for the most recent version of the project. This however does not mean that all previous versions have been uploaded, or that this is a complete version, as it might be a vulnerability. It would be good to implement a better check by retrieving all versions and checking whether the exact version is already in the database, instead of a newer one.
    \item When a job times out and the database-api removes the job, the controller will try to finish the job for a total of 6 times, because it fails every time. It would be better to not do this.
\end{itemize}



\end{document}